\section{Discussion}
This paper aims to learn the behavioral interaction of rats through the control of robotic rats. We proposed two behavior patterns in the interaction process of rats and the hypothesis of the behavioral interaction. After testing, we proved that the interaction probability between rats depends on the distance of the centroid and the average relative velocity, and obtained the optimal solution expressing their relationship. The robots learn similar movements and behaviors from rats, and then understand the interactive behavior mechanism of rats, which can significantly help us realize a more natural and effective robot-rat interaction process.

The deficiency of this study is that in the interactive control of robots, the contacting state is not considered. In future work, we will plan to simulate the contacting state of rats by increasing the DOFs of robots' forelimbs and nesting rat skins to make the robot closer to the real rats.
In this study, the rat that actively generates interaction intention is recorded as A, and the rat that passively interacts is recorded as B. In the future robot-rat interaction process, the proportion of active interaction intention (behavior) of the robot can be adjusted to play different roles (lively/quiet, strong/weak). After understanding the personality characteristics of the interaction target, the robot can make corresponding adjustments to adapt to different rat individuals to produce a better interaction effect.

This study only explored the effects of visual factors on rat interactive behavior, including centroid distance and average relative velocity. In future studies, we can add sound (ultrasound, which can reflect the emotional state of rats), smell and other factors, and give these characteristics to the robot to explore the interaction process of rats at other levels.
